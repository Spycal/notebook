\documentclass{book}
\usepackage[
  margin=1.0cm, 
  lmargin=2.5cm,
  bmargin=2cm,
  includehead,
  headheight=1cm,
  nomarginpar
]{geometry}
\usepackage{blindtext}
\usepackage{fancyhdr}
\usepackage{multicol}
\usepackage{array}
\usepackage{enumitem}
\usepackage[sfdefault]{FiraSans}
\usepackage[T1]{fontenc}
\usepackage[document]{ragged2e}
\renewcommand*\oldstylenums[1]{{\firaoldstyle #1}}
\renewcommand{\footrulewidth}{0.4pt}
\pagestyle{fancy}
\fancyhead[LO, LE]{\textbf{\LARGE{Penggajian Perniagaan: Semester 2}}}
\fancyhead[RO, RE]{\thepage}
\fancyfoot[LO, LE]{\textit{Ditulis menggunakan} \LaTeX}
\fancyfoot[CO, CE]{}
\hyphenchar\font=-1

\begin{document}
  \begin{multicols*}{2}
  [
    \section*{6.1 Pengenalan Kepada Pengurusan}
    ]
    \subsection*{Maksud Pengurusan}
      Proses formal dan sistematik melibatkan 4FP sumber manusia dan sumber bukan manusia
      
    \subsection*{Kepentingan Pengurusan}
    \begin{itemize}
      \item Membantu mencapai matlamat organisasi
      \item Mengelakkan pembaziran sumber organisasi
      \item Memotivasi modal insan
      \item Menentukan hala tuju organisasi
    \end{itemize}

    \subsection*{\underline{Fungsi Pengurusan}}
    \begin{itemize}
      \item \textbf{Perancangan} \\
        Proses menentukan VMO yang ingin dicapai dan membentuk tindakan
      \item \textbf{Pengorganisasian} \\
        Proses penyelarasan dan pengagihan sumber organisasi untuk melaksanakan tugas
      \item \textbf{Kepimpinan} \\
        Proses pengurus mencuba mempengaruhi gelagat pekerja 
      \item \textbf{Pengawalan} \\
        Proses pemantauan prestasi dan memastikan kerja akhir tercapai
    \end{itemize}
    
    \subsection*{\underline{Konsep Keberkesanan dan Kecekapan}}
    \begin{itemize}
      \item \textbf{Keberkesanan}
        \begin{itemize}[leftmargin=*]
          \item Melakukan perkara \emph{yang} betul (\textit{do the right things})
          \item Memilih kaedah tepat dalam memyempurnakan kerja/tugas
          \item Kerja/tugas dibuat mampu mencapai objektif organisasi
          \item Contohnya, Restoran berjaya memenuhi pesanan pelanggan
        \end{itemize}
      \item \textbf{Kecekapan}
        \begin{itemize}[leftmargin=*]
          \item Melakukan perkara \emph{dengan} bentuk (\textit{doing things right})
          \item Organisasi mencapai output maksimum dengan sumber minimum
          \item Kerja/tugas dibuat dapat mengurangkan kos pengeluaran dan pembaziran sumber
          \item Contohnya, Makanan restoran disiapkan lebih cepat daripada masa dijanjikan
        \end{itemize}
    \end{itemize}

    \subsection*{\underline{Peranan Pengurus}\hfill\footnotesize\emph{*menurut Teori Mintzberg}}
    \begin{enumerate}
      \item \textbf{Peranan Bermaklumat} 
      \begin{itemize}[leftmargin=*]
        \item \textbf{Sebagai pemantau atau pengawas} 
          \begin{itemize}[leftmargin=*]
            \item Pengurus memperoleh maklumat berguna daripada persekitaran dalaman/luaran organisasi
            \item Maklumat diperoleh daripada laporan, mesyuarat dan seterusnya
            \item Contohnya, memantau maklumat perniagaan untuk membuat perancangan lebih baik
          \end{itemize} 
        \item \textbf{Sebagai penyampai}
          \begin{itemize}[leftmargin=*]
            \item Pengurus yang menyampaikan maklumat kepada pekerjanya
            \item Boleh disampaikan melalui mesyuarat, memo dan media
            \item Contohnya, perancangan digariskan haruslah disampaikan kepada semua staf
          \end{itemize}
        \item \textbf{Sebagai jurucakap}
          \begin{itemize}[leftmargin=*]
            \item Pengurus yang menyampaikan maklumat kepada pihak luar/berkepentingan 
            \item Boleh disampaikan melalui ucapan, memo dan persidangan akhbar
            \item Contohnya, menyampaikan maklumat prestasi organisasi kepada media massa
          \end{itemize}
      \end{itemize}
      \item \textbf{Antara perorangan (\textit{Interpersonal})} 
      \begin{itemize}[leftmargin=*]
        \item \textbf{Sebagai lambang, simbol atau orang terpenting}
          \begin{itemize}[leftmargin=*]
            \item Pengurus menjalankan beberapa tugas berbentuk simbolik/upacara
            \item Merupakan ketua unit yang akan mewakili organisasi
            \item Contohnya, melawat pekerja yang sakit atau menghadiri jamuan tahunan
          \end{itemize}
        \item \textbf{Sebagai ketua atau pemimpin}
          \begin{itemize}[leftmargin=*]
            \item Pengurus sebagai ketua organisasi
            \item Mengarah dan menyelaraskan aktiviti pekerja demi mencapai matlamat
            \item Contohnya, memberi kata motivasi kepada pekerjanya semasa bekerja 
          \end{itemize}
        \newpage
        \item \textbf{Sebagai penghubung atau pengantara}
          \begin{itemize}[leftmargin=*]
            \item Pengurus sebagai penghubung antara organisasi dengan pihak luar organisasi
            \item Mendapatkan maklumat berkaitan persekitaran luaran
            \item Contohnya, mengetahui tentang keadaan pasaran dengan pembekal
          \end{itemize}
      \end{itemize}
      \item \textbf{Pembuatan Keputusan}
      \begin{itemize}[leftmargin=*]
        \item \textbf{Sebagai usahawan atau keusahawanan} 
          \begin{itemize}[leftmargin=*]
            \item Pengurus meningkatkan perniagaan dengan melaksanakan idea baharu
            \item Menyatukan faktor pengeluaran untuk menghasilkan keuntungan
            \item Contohnya, mencari lokasi strategi untuk membuka perniagaan
          \end{itemize}
        \item \textbf{Sebagai perunding}
          \begin{itemize}[leftmargin=*]
            \item Juru runding organisasi dengan pihak luar
            \item pihak luar seperti pembekal, pemodal, kerajaan dan lain
            \item Contohnya, juru runding dengan pembekal-pembekal yang memberi syarat terbaik
          \end{itemize}
        \item \textbf{Sebagai pembahagi sumber atau pengagih\\ sumber}
          \begin{itemize}[leftmargin=*]
            \item Mengagihkan faktor pengeluaran kepada jabatan tertentu
            \item Mengagihkan mengikut keutamaan dan kepentingan tugas kerana sumber terhad
            \item Contohnya, projek perlu diberi rungutan pelanggan yang akan diberi keutamaan
          \end{itemize}
        \item \textbf{Sebagai penyelesai masalah atau pengawal gangguan}
          \begin{itemize}[leftmargin=*]
            \item Pengurus menyelesaikan masalah dihadapi oleh pekerja
            \item Ia juga menyelesaikan konflik dalam kalangan subordinat
            \item Contohnya, menyelesaikan masalah pelanggan melanggar kontrak jual-beli 
          \end{itemize}
      \end{itemize} 
    \end{enumerate}
    
    \subsection*{\underline{Pendekatan Pemikiran Pengurusan}}
    \subsection*{Pendekatan Klasikal}
    Menekankan pengurusan kerja dan organisasi dengan bekesan
    \vfill\null\columnbreak
    \begin{enumerate}
      \item \textbf{Teori Pengurusan Saintifik} \\
        Menekankan kaedah kerja untuk meningkatkan kecekapan pekerja  
        \begin{itemize}[leftmargin=*]
          \item \textbf{Frederick Taylor}
            \begin{itemize}[leftmargin=*]
              \item Memperkenalkan pengkhususan kerja yang meningkatkan produktiviti kerja
              \item Menyarankan kajian melalui kajian  masa, pembentukan piawaian dan 
                insentif bagi prestasi pekerja
              \item \textbf{Empat Prinsip Pengurusan Saintifik}
                \begin{enumerate}[label=\roman*.]
                  \item Mengenalpasti kaedah terbaik untuk melaksanakan kerja
                  \item Memilih, melatih dan membangunkan pekerja 
                  \item Menggalakkan kerjasama antara pekerja dan pihak pengurusan
                  \item Membahagikan tugas secara adil dan saksama
                \end{enumerate}
            \end{itemize}
          \item \textbf{Frank \& Lilian Gilbreth}
            \begin{itemize}[leftmargin=*]
              \item Melakukan kajian berkaitan masa pergerakan kerja (\textit{time and motion})
              \item Menyarankan kajian untuk mengurangkan pergerakan tidak penting dalam tugasan
              \item Memfokuskan kepada aspek kemanusiaan tentang keperluan pekerja
            \end{itemize}
          \item \textbf{Henry L. Gantt}
            \begin{itemize}[leftmargin=*]
              \item Memperkenalkan kaedah penjadualan tugas melalui Carta Gantt
              \item Menekankan masa dan kos merancang dan mengawal kerja
              \item Memperkenalkan sistem ganjaran seperti bonus
            \end{itemize}
          \item \textbf{Harrington Emerson}
            \begin{itemize}[leftmargin=*]
              \item Memperkenalkan konsep kecekapan dalam menguruskan organisasi dengan cekap
              \item[\textbf{+}] Menghapuskan pembaziran dan kelemahan dalam suatu proses
              \item[\textbf{+}] Menetapkan standard prestasi untuk setiap tugas
            \end{itemize}
        \end{itemize}
        \newpage
      \item \textbf{Teori Pengurusan Pentadbiran} \\ 
        Menekankan usaha pengurus untuk mengurus organisasi secara menyeluruh
        \begin{itemize}[leftmargin=*]
          \item \textbf{Henri Fayol}
            \begin{itemize}[leftmargin=*]
              \item Orang pertama membincangkan pengurusan secara satu proses
              \item Membahagikan pengurusan iaitu perancangan, pengorganisasian, pengarahan, 
                penyelarasan dan pengawalan
              \item Memperkenalkan \textbf{14 Prinsip Asas Pengurusan}
                \begin{enumerate}[label=\arabic*, leftmargin=*]
                \item \textbf{Pembahagian kerja} \\
                  Setiap pekerja melakukan tugas khusus (\textit{pengkhususan kerja})
                \item \textbf{Esprit de corps} \\
                  Menekankan semangat berpasukan kerja dalam tugasan
                \item \textbf{Inisiatif} \\
                  Pekerja mempunyai kebebasan dalam tugasan
                \item \textbf{Ganjaran} \\
                  Pekerja diberi ganjaran yang adil dan saksama
                \item \textbf{Susunan tertib} \\
                  Peralatan dan pekerja berada tempat yang betul dan 
                  masa yang tepat
                \item \textbf{Kesamarataan/ekuiti} \\
                  Pekerja dilayan dengan adil dan saksama
                \item \textbf{Rantaian skala} \\
                  Rantaian autoriti bermula daripada peringkat atasan ke bawahan 
                \item \textbf{Disiplin}
                  Pekerja perlu mengikut arahan dan menurut prosedur
                \item \textbf{Paduan perintah}
                  Pekerja menerima arahan daripada ketua sahaja
                \item \textbf{Paduan arahan}
                  Aktiviti dilaksanakan untuk mencapai matlamat sama
                \item \textbf{Mengutamakan kepentingan organisai}
                  Kepentingan peribadi diketepikan
                \item \textbf{Pemusatan}
                  Keputusan terletak pada pengurusan atasan
                \item \textbf{Kestabilan personel}
                  Kadar keluar masuk pekerja rendah digalakkan,
                  bagi pekerja tinggi perlu dielakkan 
                \item \textbf{autoriti}
                  Memberi arahan kerja supaya tugas dilaksanakan 
              \end{enumerate}
            \end{itemize}
            \vfill\null\columnbreak
          \item \textbf{Mary Parker Follet}
            \begin{itemize}[leftmargin=*]
              \item Menyarankan pengurus menggunakan pendekatan kolaboratif dengan pekerja 
              \item Menekankan matlamat perkongsian bersama antara pengurus dengan pekerja
              \item Kumpulan informal diperlukan bagi membantu organisasi
            \end{itemize}
          \item \textbf{Chester Barnard}
            \begin{itemize}[leftmargin=*]
              \item Memperkenalkan konsep organisasi tidak formal
              \item Membantu organisasi untuk mencapai matlamat pengurusan baik
              \item Menekankan hubungan baik antara pengurus dengan pekerja
            \end{itemize}
        \end{itemize}
      \item \textbf{Teori Pengurusan Birokrasi} \\
        Menekankan autoriti dan hierarki pengurusan
        \begin{itemize}[leftmargin=*]
          \item \textbf{Max Weber}
            \begin{itemize}[leftmargin=*]
              \item Menekankan autoriti dan hierarki pengurusan
              \item Memperkenalkan TBP untuk mewujudkan kecekapan dan keberkesanan organisasi
              \item Antara prinsipnya:
                \begin{enumerate}[label=\arabic*, leftmargin=*]
                  \item Pengkhususan kerja meningkatkan kemahiran pekerja
                  \item Hierarki dan autoriti jelas supaya aktiviti dapat dikoordinasikan
                  \item Peraturan dan prosedur format mewujudkan kesamarataan
                  \item Peraturan, prosedur dan hukuman diaplikasikan secara menyeluruh dan sama rata
                  \item Pemilihan dan kenaikan pangkat berdasarkan kelayakan dan pretasi ahli
                \end{enumerate}
              \item Masalah konsep birokrasi kepada organisasi besar:
                \begin{enumerate}[label=\arabic*, leftmargin=*]
                  \item Melibatkan banyak peringkat
                  \item Memerlukan masa yang lama
                  \item Memerlukan staf yang ramai
                  \item Mementingkan proses daripada hasil kerja
                \end{enumerate}
            \end{itemize}
        \end{itemize}
    \end{enumerate}
    \newpage

    \subsection*{Pendekatan Kemanusiaan/Gelagat}
    Menekankan kepada sumber manusia dalam organisasi
    \begin{enumerate}
      \item \textbf{Pergerakan Hubungan Kemanusiaan} \\
        Menekankan hubungan dan interaksi baik antara pengurus dengan pekerja
        \begin{itemize}[leftmargin=*]
          \item \textbf{Elton Mayo} - \emph{melalui Kajian Hawthorne}
            \begin{itemize}[leftmargin=*]
              \item Menekankan aspek sosial dan pengiktirafan kumpulan kerja
              \item Produktiviti kerja dapat ditingkatkan dengan memenuhi keperluan pekerja
              \item Pengurus perlu menjaga hubungan baik antara pekerja
            \end{itemize}
          \item \textbf{Hugo Munsterberg} 
            \begin{itemize}[leftmargin=*]
              \item membincangkan cara bagaimana ahli psikologi membantu industri:
                \begin{itemize}[leftmargin=*]
                  \item Mengkaji kerja dan mengenalpasti individu sesuai dengan suatu kerja
                  \item Mengenalpasti psikologi sesuai bagi individu melakukan kerja
                  \item Membangunkan strategi untuk pekerja berkelakuan seperti dikehendaki
                \end{itemize}
            \end{itemize}
        \end{itemize}
      \item \textbf{Teori Sains Gelagat} \\ 
        Menekankan penyelidikan saintifik untuk memahami gelagat manusia
        \begin{itemize}[leftmargin=*]
          \item Teori Hierarki Keperluan Maslow (\textbf{Abraham Maslow})
            \begin{itemize}[leftmargin=*]
              \item Menyatakan manusia mempunyai lima keperluan asas dari keperluan fisiologi,
                keselamtan, sosial, penghargaan diri dan pencapaian hasrat diri
            \end{itemize}
          \item Teori X dan Teori Y (\textbf{Douglas Mc Gregor})
            \begin{itemize}[leftmargin=*]
              \item Pengurus melayan pekerjannya berdasarkan andaian beliau terhadap pekerja 
                \begin{itemize}[leftmargin=*]
                  \item \textbf{Pekerja Teori X} \\
                    Pekerja negatif seperti malas bekerja perlu dipaksa dan diarah bekerja
                  \item \textbf{Pekerja Teori Y} \\ 
                    Pekerja positif seperti rajin/minat bekerja perlu mengilap potensi pekerja
                \end{itemize}
            \end{itemize}
        \end{itemize}
    \end{enumerate}
    
    \subsection*{Pendekatan Saisn Pengurusan/Kuantitatif}
    Menekankan Pengunaan matematik, statistik dan alat bantu maklumat untuk membuat keputusan
    \vfill\null\columnbreak
    \begin{enumerate}
      \item \textbf{Teori Sains Pengurusan} \\
        Penggunaan model dan formula matematik dalam membuat keputusan pengurusan
      \item \textbf{Teori Pengurusan Operasi} \\
        Penggunaan kaedah statistik dan matematik untuk aktiviti pengurusan harian
      \item \textbf{Teori Sistem Pengurusan Maklumat} \\
        Penggunaan teknologi maklumat untuk mengurus data organisasi dan membuat keputusan pengurusan
    \end{enumerate}

    \subsection*{Pendekatan Moden/Kontemporari}
    Menekankan inovatif berkaitan pengurusan organisasi
    \begin{enumerate}
      \item \textbf{Teori Sistem}
        \begin{itemize}[leftmargin=*]
          \item Menganggap organisasi sebagai satu sistem kompleks demi mencapai matlamat bersama
          \item Tindakan dalam setiap bahagian berlainan memberi kesan antara satu sama lain 
          \item Sistem organisasi melibatkan input, proses transformasi dan output serta interaksi
          \item Contohnya, organisasi menerima input, input diubah melalui proses transformasi untuk 
            menghasilkan output
          \item \textbf{Pelopor:} \emph{Chester Barnard, Ladwig von\\ Bertalanffy dan Russell Ackoff}
        \end{itemize}
      \item \textbf{Teori Kontingensi atau Situasi}
        \begin{itemize}
          \item Tiada satu pendekatan terbaik digunakan untuk semua situasi
          \item Pengurus perlu mengenalpasti situasi dahulu sebelum memilih teknik pengurusan
          \item Gaya kepimpinan ditentukan berdasarkan situasi apabila mereka memimpin
          \item Contohnya, gaya autokratik hanya sesuai untuk pekerja berciri negatif 
          \item \textbf{Pelopor:} \emph{Tom Burns, George M. Stalker dan James Thompson}
        \end{itemize}
    \end{enumerate}
  \end{multicols*}
\end{document}
